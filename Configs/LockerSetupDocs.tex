
\documentclass[A4paper,11pt]{article}

\usepackage[margin=1.5cm]{geometry}
\usepackage{listings}
\usepackage{hyperref}
\usepackage{hypcap}

\lstset{basicstyle=\footnotesize\ttfamily,}

\title{Locker Setup Documentation V1.0}
\author{Chris Elsmore}
\date{October 2011}
\setlength{\parindent}{0pt}

\begin{document}
\maketitle

This document explains the steps to install and configure Locker and its dependencies, and to create a secure web proxy front end using Nginx, on Ubuntu Linux circa 2011 (tested on 11.04 Natty Narwhal).

\section{Locker Configuration}
\label{sec:LockerConfiguration}


\subsection{Locker Dependencies}
\label{subsec:LockerDepenedcies}

Locker uses a number of technologies, all of which need to be installed and configured before starting the locker and pulling in personal data. These include: \nameref{subsec:MongoDB}, \nameref{subsec:Node.js}, \nameref{subsec:NPM} and \nameref{subsec:Clucene} as well as the \nameref{subsec:Locker} code itself. Before starting, the Distribution should be up to date

\subsection{MongoDB}
\label{subsec:MongoDB}

Installing Mongo DB involves adding the Mongo Repositories, and then installing with apt-get.

\begin{lstlisting}
sudo apt-key adv --keyserver keyserver.ubuntu.com --recv 7F0CEB10

sudo gedit /etc/apt/sources.list
    deb http://downloads-distro.mongodb.org/repo/ubuntu-upstart dist 10gen

sudo apt-get update

sudo apt-get upgrade

sudo apt-get install mongodb-10gen
\end{lstlisting}


\subsection{Node.js}
\label{subsec:Node.js}

Installing Node.js involves first downloading and building the source. The latest Node.js that Locker currently supports is V0.4.9, and a few packages are needed before building can complete.

\begin{lstlisting}
sudo apt-get install g++ curl libssl-dev apache2-utils git-core cmake

cd ~/Downloads
mkdir nodejs && cd nodejs

wget http://nodejs.org/dist/node-v0.4.9.tar.gz
tar -xvf node-v0.4.9.tar.gz 

./configure
make -j2
sudo make install
\end{lstlisting}

\newpage
\subsection{NPM}
\label{subsec:NPM}

Installing NPM is done by running an install script that fetches and installs NPM.

\begin{lstlisting}
cd ~/Downloads
mkdir npm && cd npm

curl -O http://npmjs.org/install.sh
chmod +x install.sh
sudo ./install.sh
\end{lstlisting}

\subsection{Clucene}
\label{subsec:Clucene}

Clucene is currently built from source from the git tree, and then built and installed.

\begin{lstlisting}
cd ../
git clone git://clucene.git.sourceforge.net/gitroot/clucene/clucene .
cd Clucene
cmake -G "Unix Makefiles"
make
sudo make install
sudo ldconfig -vv | grep clucene
(clucene.sourceforge.net/download.shtml)

( hit http://localhost:8042/Me/search/update to start full re-index )

\end{lstlisting}

\subsection{Locker}
\label{subsec:Locker}

The Locker code is installed by cloning a git repo, and then installing all needed JavaScript and Python packages using NPM and Easy Install respectively.

\begin{lstlisting}
cd ~/
mkdir Locker
cd Locker
git clone https://github.com/LockerProject/Locker.git .
sudo npm install -g
    (Cluscene fails)
sudo npm install ini

sudo apt-get install python-setuptools
sudo easy_install virtualenv gdata flask pyparsing google-api-python-client

./checkEnv.sh

git submodule update --init

\end{lstlisting}

\newpage
\section{Nginx Configuration}
\label{sec:NginxConfiguration}

\begin{lstlisting}
sudo apt-get update
sudo apt-get install apache2-utils git-core curl build-essential 
sudo apt-get openssl libssl-dev nginx
cd Downloads/
git clone https://github.com/joyent/node.git && cd node
./configure
make
sudo make install
node -v

mkdir ~/Documents/multiserver-test/
mkdir ~/Documents/multiserver-test/pass
mkdir ~/Documents/multiserver-test/logs
cd ~/Documents/multiserver-test/

sudo nano /etc/nginx/proxy.conf
proxy_redirect                  off;
proxy_set_header                Host                    $host;
proxy_set_header                X-Real-IP               $remote_addr;
proxy_set_header                X-Forwarded-For         $proxy_add_x_forwarded_for;
client_max_body_size            10m;
client_body_buffer_size         128k;
proxy_connect_timeout           90;
proxy_send_timeout              90;
proxy_read_timeout              90;
proxy_buffers                   32 4k;
 => /etc/nginx/proxy.conf

sudo nano /etc/nginx/sites-enabled/nodetest
server {
	    listen	80;
	    server_name domain1.localhost;

	    access_log 	/home/elsmorian/Documents/multiserver-test/logs/domain1.access.log;
	    error_log	/home/elsmorian/Documents/multiserver-test/logs/domain1.error.log;
	
	    location / {
		    proxy_pass http://127.0.0.1:8000;
		    #include /etc/nginx/proxy.conf;
	    }
}

server {
        listen  80;
        server_name domain2.localhost;

        access_log      /home/elsmorian/Documents/multiserver-test/logs/domain2.access.log;
        error_log       /home/elsmorian/Documents/multiserver-test/logs/domain2.error.log;

        location / {
                proxy_pass http://127.0.0.1:8001;
        }
}

server {
        listen  80;
        server_name domain3.localhost;

        access_log      /home/elsmorian/Documents/multiserver-test/logs/domain3.access.log;
        error_log       /home/elsmorian/Documents/multiserver-test/logs/domain3.error.log;

        location / {
                proxy_pass http://127.0.0.1:8002;
        }
}
 => /etc/nginx/sites-enabled/nodetest

sudo invoke-rc.d nginx start

nano domain1.js
var http = require('http');
http.createServer(function(req, res) {
	res.writeHead(200, {'Content-Type': 'text/plain'});
	res.end('Hello, I\'m Domain 1, on port 8000.\n');
}).listen(8000, '127.0.0.1');
console.log('Server running locally, port: 8000');
 => domain1.js
node domain1.js

Test!

sudo nano /etc/nginx/sites-enabled/nodetest
server {
	listen	80;
	server_name domain1.localhost;

	access_log 	/home/elsmorian/Documents/multiserver-test/logs/domain1.access.log;
	error_log	/home/elsmorian/Documents/multiserver-test/logs/domain1.error.log;
	
	location / {
		proxy_pass 		http://127.0.0.1:8000;
		#include 		/etc/nginx/proxy.conf;
		auth_basichtpasswd -c -d domain2.htpasswd user2		"Restricted";
		auth_basic_user_file	/home/elsmorian/Documents/multiserver-test/pass/domain1.htpasswd;
	}
}

server {
        listen  80;
        server_name domain2.localhost;

        access_log      /home/elsmorian/Documents/multiserver-test/logs/domain2.access.log;
        error_log       /home/elsmorian/Documents/multiserver-test/logs/domain2.error.log;

        location / {
                proxy_pass http://127.0.0.1:8001;
                auth_basic              "Restricted";
                auth_basic_user_file    /home/elsmorian/Documents/multiserver-test/pass/domain2.htpasswd;
        }
}

server {
        listen  80;
        server_name domain3.localhost;

        access_log      /home/elsmorian/Documents/multiserver-test/logs/domain3.access.log;
        error_log       /home/elsmorian/Documents/multiserver-test/logs/domain3.error.log;

        location / {
                proxy_pass http://127.0.0.1:8002;
                auth_basic              "Restricted";
                auth_basic_user_file    /home/elsmorian/Documents/multiserver-test/pass/domain3.htpasswd;
        }
}
 => sudo /etc/nginx/sites-enabled/nodetest

cd pass
htpasswd -c -d domain1.htpasswd user1
htpasswd -c -d domain2.htpasswd user2
htpasswd -c -d domain3.htpasswd user3

sudo nano /etc/hosts
127.0.0.1       localhost
127.0.0.1       domain1.localhost
127.0.0.1       domain2.localhost
127.0.0.1       domain3.localhost
127.0.0.1       locker.localhost


add for Locker:
server {
        listen  80;
        server_name locker.localhost;

        access_log      /home/elsmorian/Documents/multiserver-test/logs/locker.access.log;
        error_log       /home/elsmorian/Documents/multiserver-test/logs/locker.error.log;

        location / {
                proxy_pass http://127.0.0.1:8042;
                auth_basic              "Welcome to your Locker. Please enter your username and password.";
                auth_basic_user_file    /home/elsmorian/Documents/multiserver-test/pass/locker.htpasswd;
        }
}

Certs&SSL:

mkdir certs
openssl genrsa -des3 -out server.key 1024
pass: custard
openssl req -new -key server.key -out server.csr
pass- custard
challenge pass: rainbow
mv server.key server.csr certs/
cp certs/server.key certs/server.key.org
cd certs/
openssl rsa -in server.key.org -out server.key
pass- custard
openssl x509 -req -days 365 -in server.csr -signkey server.key -out server.crt


apt-get install python-setuptools (for easy_install)

\end{lstlisting}
\end{document}
