%-----------------------------------------------------------------------------
%
%               Template for sigplanconf LaTeX Class
%
% Name:         sigplanconf-template.tex
%
% Purpose:      A template for sigplanconf.cls, which is a LaTeX 2e class
%               file for SIGPLAN conference proceedings.
%
% Guide:        Refer to "Author's Guide to the ACM SIGPLAN Class,"
%               sigplanconf-guide.pdf
%
% Author:       Paul C. Anagnostopoulos
%               Windfall Software
%               978 371-2316
%               paul@windfall.com
%
% Created:      15 February 2005
%
%-----------------------------------------------------------------------------


%\documentclass[preprint]{sigplanconf}
\documentclass[10pt,twocolumn]{sigplanconf}

% The following \documentclass options may be useful:
%
% 10pt          To set in 10-point type instead of 9-point.
% 11pt          To set in 11-point type instead of 9-point.
% authoryear    To obtain author/year citation style instead of numeric.

\usepackage{amsmath}

\begin{document}

\conferenceinfo{Eurosys '12}{April 10th, Bern, Switzerland.} 
\copyrightyear{2012} 
\copyrightdata{[to be supplied]} 

\titlebanner{banner above paper title}        % These are ignored unless
\preprintfooter{short description of paper}   % 'preprint' option specified.

\title{Title Text}
\subtitle{Subtitle Text, if any}

\authorinfo{Chris Elsmore \and Anil Mahavapeddy}
           {University of Cambridge}
           {cce25/avsm1@cl.cam.ac.uk}
\authorinfo{Name2\and Name3}
           {Affiliation2/3}
           {Email2/3}

\maketitle

\begin{abstract}
This paper discusses the problem of building an incentive scheme in a large university that wishes to optimise energy usage by encouraging behavioural change. This requires infrastructure to gather high resolution data such as employee location and device energy usage, without sacrificing the individual privacy of participants, in order to understand existing behaviour. This motivated the construction of a distributed locker scheme in which individuals record fine-grained energy, location and other data into a private data store they own, and trade portions of it to the university in return for specific incentives. We report on the hardware, software and social challenges that arose while piloting this scheme in the University of Cambridge, as well as discussing benefits compared to other possible architectures. We also identify the future advantages both to the university and it's employees and members by describing projects we are planning to run that rely heavily on such a locker or 'Personal Container'.\end{abstract}

\category{CR-number}{subcategory}{third-level}

\terms
term1, term2

\keywords
keyword1, keyword2

\section{Introduction}

The text of the paper begins here.

\appendix
\section{Appendix Title}

This is the text of the appendix, if you need one.

\acks

Acknowledgments, if needed.

% We recommend abbrvnat bibliography style.

\bibliographystyle{abbrvnat}

% The bibliography should be embedded for final submission.

\begin{thebibliography}{}
\softraggedright

\bibitem[Smith et~al.(2009)Smith, Jones]{smith02}
P. Q. Smith, and X. Y. Jones. ...reference text...

\end{thebibliography}

\end{document}
